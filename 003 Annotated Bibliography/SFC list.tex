\documentclass[twoside,a4paper,11pt]{article}
\pagestyle{plain}
\pagenumbering{arabic}
\title{A Systematic Review Categorisation of the Stock Flow literature}
%\subtitle{Working Paper 001}
\author{Graeme Smith}
% \date{date text} just print the current date.
\usepackage[authoryear]{natbib}
\usepackage{amsmath}
%\usepackage{fullpage}
\usepackage[left=2cm,top=3cm,right=2cm,bottom=3cm,bindingoffset=0.5cm]{geometry}

\begin{document}

\maketitle
\begin{abstract} This is an annotated bibliography of published papers containing Stock Flow Consistent models with a focus on those dealing with international financial imbalances, financial instability and open economies. The   reference section contains a more general listing of SFC models from the literature.
\end{abstract}

This document contains notes on a small selection of papers that are significant to my research project. Section 1 deals with papers that model international flows and imbalances. Section 2 lists models that deal with open economy issues. Section 3 looks at papers that have modelled Financial Instability, and finally Section 4 is concerned with models of the Shadow Banking System. The reference section at the end gives a fairly comprehensive, though not exhaustive, listing of publications containing Stock Flow Consistent models.

\section{SFC Models of International Financial Imbalances}
The publications in this section contain SFC models that capture international flows between two or more trading blocs or currency zones. These are considered separately from the models in section 2 which I have called `open economy' models that deal with external trade between a single economy and `the rest of the world'.

None of the models in this section deal with `gross financial flows' nor do they use real world data -- they are theoretical models that analyse behaviour based on certain theoretical assumptions.\\  
  
\begin{tabular}{lp{0.7\linewidth}}
\cite{Mazier2009b} &  J Mazier and G Tiou Tagba Aliti. \textit{World imbalances, exchange
rates and macroeconomic adjustments: a three countries model stock  flow consistent model.} In Congr\`es AFSE, page 37, 2009. 
\end{tabular}\\[5pt]
This paper deals with macroeconomic imbalances in a three country model with the US, China and the EU, however it does not consider gross financial flows. It is concerned with effects on the current account imbalances arising in two comparable exchange rate regimes.  It proposes two models, the first with a fixed dollar-yuan parity, the second with a flexible parity. In the first scenario, `supply shocks' have a significant efffect on world imbalances and only the dollar-euro rate is able to absorb them. In the second scenario, the flexible rate provides a powerful adjustment mechanism and imbalances are significantly reduced.\\
The paper is cited by two others:
\begin{itemize}
\item[] \cite{Lavoie2010a}
\item[] A paper in French dealing with Eurozone imbalances, it does not use SFC modelling.
\end{itemize}
\centering \rule{5cm}{1pt} \\[1cm]

\raggedright \begin{tabular}{lp{0.7\linewidth}}
\cite{Mazier2012b} & J Mazier and G Tiou Tagba Aliti. World Imbalances And Macroeconomic Adjustments:
A Three Country Stock Flow Consistent Model With Fixed Or Flexible Prices.
Metroeconomica, 63(2):358-388, 2012.
\end{tabular}\\[5pt]
This paper is a later update (2012)  on the one above by the same authors and was published in Metroeconomica. It adds a third exchange rate scenario, a generalization of the two others with flexible prices rather than constant prices, presumably to answer the contention from mainstream macro that price adjustment will restore equilibrium. Their findings are that the difference between fixed and flexible prices was not barely significant compared to the differences in exchange rate regimes.\\
The paper is cited by eight others:
\begin{itemize}
\item[] \cite{Caverzasi2013}
\item[] \cite{Caverzasi2014a}
\item[] \cite{Benassy-Quere2013}
\item[] \cite{Mazier2015}
\item[] \cite{Kinsella2012a}
\item[] \cite{Valdecantos2015b}
\item[] \cite{Halporn2012}
\item[] and one other that is not concerned with SFC modelling.
\end{itemize}
\centering \rule{5cm}{1pt}  \\[1cm]

\raggedright \begin{tabular}{lp{0.7\linewidth}}
\cite{Aliti2012a}  & G Tiou Tagba Aliti and Stephen Kinsella. Simulating Global Rebalancing Using a
Yuan Anchor in a Currency Basket with a Four-Country Stock-Flow Consistent Model.
Social Science Research Network (SSRN), 2012.
\end{tabular}\\[5pt]
Like those described above, this paper uses a SFC model to consider the effects on current account imbalances of various exchange rate regimes. It uses a four sector model with the dollar, the euro, the renmimbi and a basket of currencies to represent the `rest of the world'. As with the models above, it examines the effect on current account imbalances of competitveness shocks. They find that the extent of the adjustments is sensitive to the weighting of the currencies in the basket.\\[5pt]
The paper currently has no citations listed in Google Scholar.\\
\centering \rule{5cm}{1pt}  \\[1cm]

\raggedright \begin{tabular}{lp{0.7\linewidth}}
\cite{Lavoie2010a}  &  M Lavoie and J Zhao. A Study Of The Diversifcation Of China's Foreign Reserves
Within A Three Country Stock Flow Consistent Model. Metroeconomica, 61(3):558-592, 2010.
\end{tabular}\\[5pt]
This paper presents a three currency model with the US dollar, the euro and the renmimbi. It simulates the effect of a diversification of China's foreign currency reserves away from the dollar and into the euro. The simulation results show that China and the USA both benefit from diversification, while the Euro economy slows down. Interestingly, the model generates path dependence \\[5pt]
The paper is cited by 17 others:
\begin{itemize}
\item[] \cite{Caverzasi2013}
\item[] \cite{Caverzasi2014a}
\item[] \cite{Mazier2012b} 
\item[] \cite{Mazier2013a}
\item[] \cite{Mazier2015}
\item[] \cite{Kinsella2012a}
\item[] \cite{Valdecantos2015b}
\item[] \cite{Halporn2012}
\item[] and some others that are not of interest.
\end{itemize}
\centering \rule{5cm}{1pt}  \\[1cm]

\raggedright \begin{tabular}{lp{0.5\linewidth}}
\cite{Benassy-Quere2013}  & Agn\`es B\'enassy-Qu\'er\'e and Yeganeh Forouheshfar. The Impact of Yuan Internationalization on the Euro-Dollar Exchange Rate. Mar 2013. URL http://papers.ssrn.com/abstract=2232680.
\end{tabular}\\[5pt]
This paper uses two models -- a static model and a dynamic SFC model -- of a three currency world (dollar, euro, yuan) to analyse the effect of the internationalisation of the yuan (represented by a rise in the yuan in international portfolios) on the euro-dollar exchange rate. The static model shows that the effect would be neutral whereas the dynamic model shows that exchange rate variations would be more stabilising on net foreign asset positions.  \\[5pt]
The paper is cited by four others none of which is relevant to SFC modelling.\\
\centering \rule{5cm}{1pt}  \\[1cm]

\raggedright \begin{tabular}{lp{0.7\linewidth}}
\cite{Halporn2012} & Sebastian V. Halporn and Gennaro Zezza. Reforming the International Monetary System.
A stock-flow-consistent approach. In Speculation and regulation in commodity
markets: The Keynesian approach in theory and practice, chapter 11. Reform, pages 243-272. 2012.
\end{tabular}\\[5pt]
This paper presents a SFC model for the analysis of alternative organizations of the international monetary system, and reports some simulation results from the model. The model has four blocks: the United States, the Euro zone; China and the ‘Rest of the World’. All countries trade with each other in both goods and financial assets. The model is able to replicate the ‘Triffin dilemma’: a restrictive monetary policy in the country issuing the international currency will have global recessionary effects. Simulations of an International Clearing Union as proposed by Keynes in the Bretton Woods conference is shown to automatically adjust current account imbalances. \\[5pt]
The paper currently has no citations listed in Google Scholar.\\
\centering \rule{5cm}{1pt}  \\[1cm]

\raggedright \begin{tabular}{lp{0.7\linewidth}}
\cite{Valdecantos2015b}  & Sebastian Valdecantos and Gennaro Zezza. Reforming the international monetary system: a stock flow-consistent approach. Journal of Post Keynesian Economics, 38(2):167-191, Oct 2015.
\end{tabular}\\[5pt]
This paper views the global macroeconomic imbalances from  the viewpoint of the international financial system and constructs SFC models to model various reform scenarios to gauge their effectiveness in reducing imbalances as well as avoiding the recessionary bias that is built into the current arrangements. They divide the world into four blocs -- US, EZ, China and ROW. The latter two fix their exchange rates against the dollar, the euro-dollar rate is floating. Each block trades with the others in goods and services and financial assets. Each block is modelled  as a five sector model -- households, non-financial firms, the financial sector, cental bank and government. They then investigate effects on imbalances and global output using three different model closures, the first with the US dollar as the international medium of exchange, the second with an increased role for SDRs and finally a bancor model. In the bancor model, global imbalances are eliminated.\\[5pt]
This paper currently has no citations listed in Google Scholar.\\
\centering \rule{5cm}{1pt}  \\[1cm]


\raggedright \begin{tabular}{lp{0.7\linewidth}}
\cite{Belabed2014} & Belabed, C.A., Theobald, T. and van Treeck, T., 2013. Income Distribution and Current Account Imbalances, Berlin. Available at: http://www.econstor.eu/handle/10419/105993.
\end{tabular}\\[5pt]
A three-country, stock-flow consistent macroeconomic model to study the effects of changes in both personal and functional income distribution on national current account balances. It develops the hypothesis of income inequality as a contributor to current account imbalances.\\[5pt]
The paper is cited by 16 others.\\
\centering \rule{5cm}{1pt}  \\[1cm]

\raggedright \section{SFC Models of Open Economies}
\begin{tabular}{lp{0.7\linewidth}}
\cite{Caldentey2007}  & Esteban P Caldentey. Balance of payments constrained growth within a consistent stock flow framework: An application to the economies of CARICOM. Technical report, ECLAC, 2007.
\end{tabular}\\[5pt]
This paper currently has no citations listed in Google Scholar.\\
\centering \rule{5cm}{1pt}  \\[1cm]

\raggedright \begin{tabular}{lp{0.6\linewidth}}
\cite{DosSantos2010a} & Claudio H. Dos Santos and Antonio Carlos Macedo e Silva. Revisiting `New Cambridge': The Three Financial Balances in a General Stock-Flow Consistent Applied Modeling Strategy. 2010.
\end{tabular}\\[5pt]
Uses the `New Cambridge'  three balances (private, public, external) to introduce open economy issues into a stock flow consistent model.\\[5pt]
The paper is cited by 8 others:
\begin{itemize}
\item[] \cite{Cripps2011}
\item[] \cite{Shaikh2012}
\item[] \cite{Dafermos2015a}
\item[] \cite{Leite2015}
\item[] \cite{Michell2012}
\item[] and three others that are not published in English.
\end{itemize}
\centering \rule{5cm}{1pt}  \\[1cm]

\raggedright \section{SFC Models of  Financial Instability}
\begin{tabular}{lp{0.7\linewidth}}
\cite{Passarella2011a} & Passarella, M., 2011. Systemic financial fragility and the monetary circuit: a stock-flow consistent approach, Bergamo. Available at: http://mpra.ub.uni-muenchen.de/28498/ [Accessed October 8, 2015].
\end{tabular}\\[5pt]
\centering \rule{5cm}{1pt}  \\[1cm]

\raggedright\begin{tabular}{lp{0.7\linewidth}}
\cite{Kinsella2011} & Kinsella, S., 2011. Words to the Wise: Stock Flow Consistent Modeling of Financial Instability. Geary Institute, University College Dublin, Working Papers: 201130, 2011, 14 pp., p.14.
\end{tabular}\\[5pt]
\centering \rule{5cm}{1pt}  \\[1cm]

\raggedright\begin{tabular}{lp{0.7\linewidth}}
\cite{LeHeron2009a} & Le Heron, E., 2009. Financial Crisis And Banking Behaviour In A Post-Keynesian Stock-Flow Consistent Model, CEPN - UMR 7115, Paris.
\end{tabular}\\[5pt]
\centering \rule{5cm}{1pt}  \\[1cm]

\raggedright\begin{tabular}{lp{0.7\linewidth}}
\cite{Tymoigne2006a} & Temoigne, E., 2006. System dynamics modeling of a stock flow consistent Minskyan model/by Eric Temoigne (The Minskyan system.
\end{tabular}\\[5pt]
\centering \rule{5cm}{1pt}  \\[1cm]

\raggedright\begin{tabular}{lp{0.7\linewidth}}
\cite{Michell2014d} & Michell, J., 2014. Speculation, financial fragility and stock-flow consistency. In R. Bellofiore \& G. Vertova, eds. The Great Recession and the Contradictions of Contemporary Capitalism. Cheltenham, UK: Edward Elgar.
\end{tabular}\\[5pt]
\centering \rule{5cm}{1pt}  \\[1cm]

\raggedright\begin{tabular}{lp{0.7\linewidth}}
\cite{DosSantos2004} & Dos Santos, C.H., 2004. A stock-flow consistent general framework for formal Minskyan analyses of closed economies, Annandale-on-Hudson, NY.
\end{tabular}\\[5pt]
\centering \rule{5cm}{1pt}  \\[1cm]

\raggedright\begin{tabular}{lp{0.7\linewidth}}
\cite{Mouakil2014} & Mouakil, T., 2014. A “Minsky crisis” in a Stock-Flow Consistent model. Revue de la régulation. Capitalisme, institutions, Pouvoirs, 16. Available at: http://regulation.revues.org/10963 [Accessed October 8, 2015].
\end{tabular}\\[5pt]
\centering \rule{5cm}{1pt}  \\[1cm]



\raggedright \section{SFC Models of  Shadow Banking}
\begin{tabular}{lp{0.7\linewidth}}
\cite{Miess2015} & Miess, M.G. \& Schmelzer, S., 2015. A Stock-flow Consistent Model of Financialisation and Shadow Banking: Financial Fragility in a Modern Capitalist Economy. In IMK FMM Conference, Berlin 2015.
\end{tabular}\\[5pt]
\centering \rule{5cm}{1pt}  \\[1cm]

\raggedright\begin{tabular}{lp{0.7\linewidth}}
\cite{Cao2015} & Cao, S., 2015. A Stock-Flow Consistent Model of the Shadow Banking System with Some Minsky Dynamics. University of Ottawa.
\end{tabular}\\[5pt]
\centering \rule{5cm}{1pt}  \\[1cm]


\nocite{*}

\bibliographystyle{plainnat}
\bibliography{sfc}
\end{document}
