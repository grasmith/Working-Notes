\documentclass[twoside,a4paper,11pt]{article}
\pagestyle{plain}
\pagenumbering{arabic}
\title{The Core Model and Sources of Data}
%\subtitle{Working Paper 001}
\author{Graeme Smith}
% \date{date text} just print the current date.
\usepackage[authoryear]{natbib}
\usepackage{amsmath}
\usepackage[left=2cm,top=3cm,right=2cm,bottom=3cm,bindingoffset=0.5cm]{geometry}
\begin{document}

\maketitle
\begin{abstract} This note presents an outline of a Stock Flow Consistent (SFC) model to investigate the connection between global economic imbalances and global financial instability. The post-crisis financial literature reveals various views on the contribution of global imbalances to the financial crisis. A first group is concerned with the large current account imbalances that developed in the pre-crisis period especially in the US. Prominent in this group are the `global savings glut' view \cite{Bernanke2005},  the ‘excess demand for safe assets’ view \cite{Caballero2006} and the ‘Bretton Woods II’ view \cite{Dooley2003}. A second group gives more attention to the large international financial flows that built up in this period from which has emerged the `excess financial elasticity' view \cite{Borio2011b}. It is this view which underlies the model presented here.
\end{abstract}

The purpose of this note is to test the availability of economic and financial data to populate and calibrate an empirical model capturing the global financial flows in the pre-crisis period. Section 1 outlines the theory behind the model as set out by various authors in the financial literature following the crisis. Section 2 presents a first iteration of a model that would capture these flows. Section 3 lists publicly available data sources that could be used to populate the model.

\section{The Theory}
The Excess Financial Elasticity View \cite{Borio2011b} shifts attention  from current account balances to the gross financing flows that are generated by international economic activity and whether the global economy can prevent the overall expansion of credit which contributes to the unsustainable build-up of financial imbalances. By financial imbalances they mean overinflated balance sheets driven by rapid increases in credit and asset prices which then drive unsustainable expenditure. They focus on the monetary regimes that set monetary conditions in the various currencies, the financial regimes that set constraints on financial intermediation in the various national jurisdictions, and on the interaction between the two.

\cite{Bertaut2012} also argue that the global savings glut view is incomplete as an explanation of global  financial  imbalances in this period. Because the GSG (Global Savings Glut) countries (by which they mean the Asian countries following an export-driven growth model) for the most part restricted their U.S. purchases to Treasuries and Agency debt, their provision of savings to the risky subprime mortgage borrowers was indirect, pushing down yields on safe assets, and driving a `search for yield' on the part of other investors increasing their appetite for alternative investments. A more complete picture of how capital flows contributed to the crisis must pay attention to the large inflows from European countries, most of which went into asset-backed securities (ABS), including mortgage-backed securities and other structured investment products rather than `safe assets' (treasuries and agency debt). The current account position of these European countries was roughly in balance overall, with some running a significant surplus (Germany, Netherlands) and others with significant deficits (UK). Hence the current account position was not a crucial factor in understanding the global imbalances.

Further support for this view comes from \cite{Acharya2010} by analyzing the geography of `financial conduits' set up by large commercial banks. They show that  banks located in both surplus countries and deficit countries manufactured `riskless assets' by selling short-term asset-backed commercial paper to risk-averse investors, predominantly US money market funds, and investing the proceeds primarily in long term US assets (mainly ABS). Many of these conduits were sponsored by European banks while the short-term funding (usually 30 days or less) was coming from the US. When the credit quality of these securities became known in 2007, the `conduits'  were unable to roll over the short-term funding and the sponsoring banks had to provide a liquidity backstop  \cite{Baba2009} which led to a global `dollar shortage' once the crisis began \cite{McGuire2009}. \cite{Brender2010} discuss international `risk takiing chains' that recycled some of the GSG funds into riskier US investments. This model will these international flows and their impact on the financial instability of the US economy.

In credit booms, asset values rise, improving balance sheets and facilitating the further expansion of credit. As a result, subsequent collapses are all the more traumatic. (The carry trade involves similar dynamics.) A capital inflow episode likewise may strengthen financial sector assets and even the NIIP in the receiving country in a way that pushes domestic borrowing beyond the point of true sustainability. This often sets the stage for a disorderly collapse later on. In diagnosing such situations, it is essential to keep the underlying credit flows in clear view  Journal of International Money and Finance 31 (2012) 469–480 p 479.

\section{The Model}
Technically, an SFC model is formulated as a dynamic system of difference equations.
The behaviour of the system depends on the functional relations between the
variables (i.e. the model equations), the parameter values, and the initial conditions. A
model can either be unstable with the value of the variables in time going to infinity
(diverging behavior), or converge to a stationary state or a limit cycle (converging behaviour). There may also be the possibility of chaotic behaviour. A particular case of divergent behavior has been termed a `steady state' \cite[:p9]{Caverzasi2014a} where stocks and flows are growing at a constant rate.

Most models depicting theoretical phenomena are calibrated to a stationary state,
which is used as a basis for policy experiments or the analysis of shocks.  The calibration procedure fixes values of the equation parameters  to ensure that model variables will fit the observed data. In a similar style,  
This model, by contrast, will be a fully empirical one in the manner discussed in \cite{Kinsella2012b}.

The model will be centred on the US economy with  three foreign blocs -- China, which will act as a proxy for the global savings glut (GSG) countries, Europe (the eurozone + UK) which will capture the investment-driven finanial flows and the rest of the world (ROW) which will be a residual to preserve the adding up constraints. The model will use changes in national balance sheets to capture the effects of international financial flows on the financial stability of the individual economies. Table 1 shows the balance sheet of the model. The sectors will be households (HH), non-financial firms (NFF), Banks (B), Other Financial Institutions (OFI), the Government (G), the central bank (CB), but not all sectors will be present in each bloc.

The assets are 
\begin{itemize}
\item[] Real Assets, which includes household real estate
\item[] Deposits
\item[] Equities
\item[] Bank Loans, including household mortgages
\item[] Treasuries and Agencies, these are the main classes of government debt. 
\item[] Money Market Mutual Funds (MMMF)
\item[] Securities, asset-backed securities and corporate bonds
\item[] CB advances
\item[] Reserves
\end{itemize}

\begin{table}[h]
\begin{center}
\begin{tabular}{llllllll}
\hline
& HH & NFF & B & OFI & G & CB & $\Sigma$  \\
\hline
Real Assets &  &  &  &  &  &  & 0 \\
Deposits &  &  &  &  &  &  & 0 \\
Equities  &  &  &  &  &  &  & 0 \\
Bank Loans &  &  &  &  &  &  & 0 \\
T Bills  &  &  &  &  &  &  & 0 \\
MMMF &  &  &  &  &  &  & 0 \\
Securities  &  &  &  &  &  &  & 0 \\
CB advances  &  &  &  &  &  &  & 0 \\
Reserves  &  &  &  &  &  &  & 0 \\
Net Worth  &  &  &  &  &  &  & 0 \\
\hline
\end{tabular}
\end{center}
\caption{The Balance Sheet for a typical trading bloc}
\end{table}

\section{The Data Sources}
One of the reasons for centering the model on the USA is the ready availability of US data, notably  from the US Flow of Funds maintained by the Federal Reserve Boards.   
\subsection{The United States}
The Federal Reserve Board's statistical release Z.1 is  usually referred to as the `Flow of Funds accounts (FoF) or the ``Financial Accounts of the United States'' to give it its full name. It consists of the Flow of Funds, Balance Sheets, and Integrated Macroeconomic Accounts. The flows in these accounts are normally expressed as net flows. To separate them out into gross flows, the Treasury International Capital (TIC) system provides detailed data on the composition of U.S. capital flows and the U.S. external position by country and instrument. In addition, the Bank of International Settlements (BIS) data on international banking positions, and the IMF’s Coordinated Portfolio Investment Survey (CPIS), provides geographic breakdowns of many countries’ external securities claims.

The FoF provides the following information:
\begin{itemize}
\item[] Matrices summarizing stocks and flows across sectors, tables on debt growth, net national wealth, gross domestic product (GDP), national income, saving, and so on
\item[] Flows of financial assets and liabilities, by sector and by financial instrument
\item[] Stocks of financial assets and liabilities, by sector and by financial instrument
\item[] Balance sheets, including nonfinancial assets, and changes in net worth for households and nonprofit organizations, nonfinancial corporate businesses, and nonfinancial noncorporate businesses
\item[] The Integrated Macroeconomic Accounts (IMA). These relate production, income, saving, and capital formation from the national income and product accounts (NIPA) to changes in net worth from the “Financial Accounts” on a sector-by-sector basis. The IMA  are based on international guidelines and terminology defined in the System of National Accounts (SNA2008).
\end{itemize}
The sectors include:
\begin{itemize}
\item[] Households and Nonprofit Institutions Serving Households (NPISH)
\item[] Nonfinancial Noncorporate Businesses
\item[] Nonfinancial Corporate Businesses
\item[] Financial Businesses
\item[] Federal Government
\item[] State and Local Governments
\item[] Rest of the World
\end{itemize}
For the purposes of this study the two non-financial sectors will be merged to give the NFF sector, state and local governments will be merged with the federal government to give sector G and data for China and the EU will be disaggregated from the Rest of the World sector. There are further detailed tables that further disaggregate certain sectors, e.g. the monetary authority is separated out in tables F.109 and L.109 from which the sector CB can be formed. Other detailed tables give a breakdown of the Financial sector from which the MMMF sector can be formed. 

%Particularly important tables will be:
%\begin{itemize}
%\item[] D.1 Debt Growth by Sector
%\item[] B.1 Derivation of U.S. Net Wealth
%\item[] F.4 Saving and Investment by Sector
%\item[] F.6 Derivation of Measures of Personal Saving
%\item[] L.6 Assets and Liabilities of the Personal Sector
%\item[] 
%\item[] 
%\end{itemize}

\subsection{The EU}
\subsubsection{The Eurozone}
Eurostat accounts: national income accounts and balance of payments for all EU countries.

ECB data:
The Statistical Data Warehouse (http://sdw.ecb.europa.eu/reports.do). This includes the economics bullsting and the statistics bulletin which consists of the following sections:\\
\begin{quote}
Euro area overview S 5\\
1 Monetary policy statistics S 6\\
2 Money, banking and other financial corporations S 10\\
3 Euro area accounts S 26\\
4 Financial markets S 34\\
5 Prices, output, demand and labour markets S 46\\
6 Government finance S 55\\
7 External transactions and positions S 60\\
8 Exchange rates S 72\\
9 Developments outside the euro area S 74\\
Notes S 76\\
\end{quote}
Section 7 provides portfolio flows and investments in financial derivatives.

\subsubsection{UK}
UK National Income Accounts:\\
The Pink Book:\\
The Blue Book:

\subsection{China}
TBD

\subsection{Data Disaggregation}
The flows in these accounts are normally net flows. To split them out into gross flows additional, more-detailed data will be required. These sources include: 1) additional breakdowns that are provided in the US Flow of Funds accounts, as well as the euro area and U.K. Balance of Payments data; 2) BIS locational data, in which the Bank of International Settlements splits banks’ cross-border positions into those with other banks and those with non-banks; and 3) aggregate balance sheet data published by the euro area and the U.K. for banks, other financial firms, and non-financial firms.

%\subsection{Bank of International Settlements (BIS)}

\bibliographystyle{plainnat}
\bibliography{WN002}
\end{document}
