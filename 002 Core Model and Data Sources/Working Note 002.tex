\documentclass[twoside,a4paper,11pt]{article}
\pagestyle{plain}
\pagenumbering{arabic}
\title{The Core Model and Sources of Data}
%\subtitle{Working Paper 001}
\author{Graeme Smith}
% \date{date text} just print the current date.
\usepackage[authoryear]{natbib}
\usepackage{amsmath}

\begin{document}

\maketitle
\begin{abstract} This note presents an outline of a Stock Flow Consistent (SFC) model to investigate the connection between global economic imbalances and the global financial crisis. The post-crisis financial literature reveals various views on the contribution of global imbalances to the financial crisis. A first group is concerned with the large current account imbalances that developed in the pre-crisis period especially in the US. Prominent in this group are the `global savings glut' view \cite{Bernanke2005},  the ‘excess demand for safe assets’ view \cite{Caballero2006} and the ‘Bretton Woods II’ view \cite{Dooley2003}. A second group gives more attention to the large international financial flows that built up in this period from which has emerged the `excess financial elasticity' view \cite{Borio2011b}. It is this view which underlies the model presented here.
\end{abstract}

The purpose of this note is to test the availability of economic and financial data to populate and calibrate an empirical model capturing the global financial flows in the pre-crisis period. Section 1 outlines the theory behind the model as set out by various authors in the financial literature following the crisis. Section 2 presents a first iteratiion of a model that would capture these flows. Section 3 lists publicly available data sources that could be used to populate the model.

\section{The Theory}
The Excess Financial Elasticity View \cite{Borio2011b} shifts attention  from current account balances to the gross financing flows that are generated by international economic activity and whether the global economy can prevent the overall expansion of credit which contributes to the unsustainable build-up of financial imbalances. By financial imbalances they mean overinflated balance sheets driven by rapid increases in credit and asset prices which then drive unsustainable expenditure. They focus on the monetary regimes that set monetary conditions in the various currencies, the financial regimes that set constraints on financial intermediation in the various national jurisdictions, and on the interaction between the two.

\cite{Bertaut2012} also argue that the global savings glut view is incomplete as an explanation of global  financial  imbalances in this period. Because the GSG (Global Savings Glut) countries (by which they mean the Asian countries following an export-driven growth model) for the most part restricted their U.S. purchases to Treasuries and Agency debt, their provision of savings to the risky subprime mortgage borrowers was indirect, pushing down yields on safe assets, and driving a `search for yield' on the part of other investors increasing their appetite for alternative investments. A more complete picture of how capital flows contributed to the crisis must pay attention to the large inflows from European countries, most of which went into asset-backed securities (ABS), including mortgage-backed securities and other structured investment products rather than `safe assets' (treasuries and agency debt). The current account position of these European countries was roughly in balance overall, with some running a significant surplus (Germany, Netherlands) and others with significant deficits (UK). Hence the current account position was not a crucial factor in understanding the global imbalances.

Further support for this view comes from \cite{Acharya2010a} by analyzing the geography of `financial conduits' set up by large commercial banks. They show that  banks located in both surplus countries and deficit countries manufactured `riskless assets' by selling short-term asset-backed commercial paper to risk-averse investors, predominantly US money market funds, and investing the proceeds primarily in long term US assets (mainly ABS). Many of these conduits were sponsored by European banks while the short-term funding (usually 30 days or less) was coming from the US. When the credit quality of these securities became known in 2007, the `conduits'  were unable to roll over the short-term funding and the sponsoring banks had to provide a liquidity backstop which led to a global `dollar shortage' once the crisis began \cite{}
Baba,McCauley, and Ramaswamy (2009) and McGuire and von Peter (2009).

Brender and Pisani (2010) discuss international `risk takiing chains' that recycled some of the GSG funds into riskier US investments , Caballero (2009) excess demand for safe assets. Johnson (2009)

\section{The Model}
The countries involved can initially be divided into three blocs - the US, the GSG countries and the EU countries.   
Table 1 shows the balance sheet of the model.
\begin{center}
\begin{tabular}{lllll}
\hline
& Households & Production & Government & $\Sigma$  \\
\hline
Money Stock & +H & 0 & -H & 0 \\
\hline
\end{tabular}
\end{center}

The transactions matrix is shown in Table 2. The first five rows contain the variables that correspond to components of the National Income accounts arranged as transactions between sectors and which take place in some specific time interval such as a quarter or a year. Row 6 expresses how the stocks are updated by the flows of the period.
\begin{center}
\begin{tabular}{lllll}
\hline
& Households & Production & Government & $\Sigma$ \\
\hline
1. Consumption & $-C_d$ & $+C_s$ &  & 0 \\
2. Govt Expenditure &  & $+G_s$ & $-G_d$  & 0 \\
3. [Output] &  & [Y] &  & 0 \\
4. Factor Income (Wages)  & $+W.N_s$ & $-W.N_d$ &   & 0 \\
5. Taxes & $-T_s$ &  & $+T_d$  & 0 \\
6. Change in the Money Stock & $-\Delta H_h$ &  & $+\Delta H_s$   & 0 \\
\hline
$\Sigma$  & 0 & 0 & 0 & 0 \\
\hline
\end{tabular}
\end{center}

The subscripts $s$, $d$, denote quantities supplied and demanded; The subscript $h$ on the money stock indicates the stock of money held by households, and in this simple model, this is balanced by the amount of money supplied by the government.

Since each row and column sums to zero, there is an \lq adding up constraint' that gives rise to an equation or identity for each one. Column one gives rise to the household budget constraint, wage income less consumption less taxes is equal to the change in money holdings.
$$W.N_s - C_d -T_s = \Delta H_h$$
Column two says that output $Y$ is equal to expenditure (consumption plus government expenditure) and is also equal to income which is  total wages in a labour only economy. Wages are the product of the wage rate $W$ and the number employed $N$.
$$Y = C_s + G_s = W.N_d$$
Column three states that the government's deficit (or surplus) is balanced by the change in the money supply.
$$ G_d -T_d = \Delta H_s $$
The horizontal rows also generate adding up constraints which express the requirement of equality of supply and demand in the various transactions:
\begin{align*}
C_s & = C_d\\
G_s & = G_d\\
N_s & = N_d\\
T_s & = T_d\\
\Delta H_h & = \Delta H_s
\end{align*}
While these constraints must hold for the model to be consistent, they don't contain any information about how these flows are to be brought into equality. To complete the model requires additional equations capturing the relationships between the flow variables. The following two behavioural equations, the tax schedule and the consumption function, will close the model:
\begin{align*}
T_s & = \theta.W.N_S\\
\intertext{and,}
C_d & = \alpha_1.YD + \alpha_2.H_{h-1}\\
\intertext{where $\theta$ is the tax rate, $\alpha_1$ is the propensity to consume out of income, $\alpha_2$ is the propensity to consume out of (last period's) wealth and $YD$ is disposable income,}
YD & = W.N_s -T_s
\end{align*}

The end result is a system of equations with a set of twelve endogenous variables
$\{C_s, C_d, G_s, G_d, N_s, N_d, T_s, T_d, H_h, H_s, Y, YD\}$ (with two of the endogenous variables having lags, $H_h, H_s$); a set of three exogenous variables $\{W, G_s, G_d\}$, 
and three parameters, $\{\theta, \alpha_1, \alpha_2 \}$.

\section{The Data Sources}
\subsection{USA:}

\subsection{UK:}
UK National Income Accounts:
The Pink Book:
The Blue Book:


\bibliographystyle{alpha}
\bibliography{WN002}
\end{document}
