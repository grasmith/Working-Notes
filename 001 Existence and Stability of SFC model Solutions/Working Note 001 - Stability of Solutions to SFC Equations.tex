\documentclass[twoside,a4paper,11pt]{article}
\pagestyle{plain}
\pagenumbering{arabic}
\title{Existence and Stability of Solutions to SFC Models}
%\subtitle{Working Paper 001}
\author{Graeme Smith}
% \date{date text} just print the current date.
\usepackage[authoryear]{natbib}
\usepackage{amsmath}

\begin{document}

\maketitle
\begin{abstract} This note examines conditions for the existence of solutions to systems of equations in Stock Flow Consistent models. These models consist of endogenous variables, exogenous variables and parameters which are linked by equations and accounting  identities. In larger models, the number of equations can be large. The equations are mostly linear but may also contain non-linear elements. The questions this note addresses concern the existence and stability of solutions to these systems of equations. 
\end{abstract}

Stock flow consistent models are based on the principle that stocks generate flows and flows update stocks. Starting from a particular state (stock value), the transactions of the period (flows) will lead to a new state (updated stock values). They are 'stock flow consistent' since, in any state, the values of the flows can be derived from the stocks and vice versa. The models are constructed by identifying the stocks (assets) that are relevant to the problem --- the balance sheet --- and the transactions related to those assets. As a further dimension, the models are sectoral, an economy is disaggregated into sectors which transact with each other --- typically households, firms, government and the financial sector. This structure can be represented by sets of equations which capture the consistency constraints and the transactional relationships between the sectors. The equations are inherently dynamic since the stock values in any period are derived from previous stock values updated by the transactional flows. So, given an initial state, the transactional relationships will generate a new state and repeated application of this will generate a time series of values of the stock and flow variables.

\cite[pg.~71]{godleylavoie:2007} distinguish between a \emph{steady state} solution, where the key variables (both stocks and flows) remain in a constant relationship to each other, and a \emph{stationary state} where, in addition, the \emph{levels} of the variable are constant. In general, the steady state will be a growing economy where \emph{ratios} of variables will be constant.

This leads to a number of basic questions:
\renewcommand{\labelitemi}{\hspace{1cm}}
\renewcommand{\labelitemii}{\hspace{1cm}}
\begin{itemize}
\item Will a given system of equations have a \lq solution', i.e. will the time series converge to a stable \lq equilibrium' state?
\item If so,
   \begin{itemize}
   \item Is this solution unique, or are there possibly multiple solutions for a given initial state and set of equation parameters?
   \item Is this solution stable, i.e. if there is a \lq shock' to the system does it return to the original state? Or does it converge to a new stable state? Or are there conditions where it becomes unstable --- oscillatory, or divergent?
   \end{itemize}
\end{itemize}
 
This note is in three sections, the first presents the model to be analysed, the second addresses the question of the long-run state and the third looks at stability conditions. 

\section{The Model SIM}
In this section, the model SIM from \cite[chap.~3]{godleylavoie:2007} is taken as an example. This is the simplest possible model with only a single asset (cash) and three sectors --- households, production and government --- it is a labour only economy.

Table 1 shows the balance sheet of model SIM.
\begin{center}
\begin{tabular}{lllll}
\hline
& Households & Production & Government & $\Sigma$  \\
\hline
Money Stock & +H & 0 & -H & 0 \\
\hline
\end{tabular}
\end{center}

The transactions matrix is shown in Table 2. The first five rows contain the variables that correspond to components of the National Income accounts arranged as transactions between sectors and which take place in some specific time interval such as a quarter or a year. Row 6 expresses how the stocks are updated by the flows of the period.
\begin{center}
\begin{tabular}{lllll}
\hline
& Households & Production & Government & $\Sigma$ \\
\hline
1. Consumption & $-C_d$ & $+C_s$ &  & 0 \\
2. Govt Expenditure &  & $+G_s$ & $-G_d$  & 0 \\
3. [Output] &  & [Y] &  & 0 \\
4. Factor Income (Wages)  & $+W.N_s$ & $-W.N_d$ &   & 0 \\
5. Taxes & $-T_s$ &  & $+T_d$  & 0 \\
6. Change in the Money Stock & $-\Delta H_h$ &  & $+\Delta H_s$   & 0 \\
\hline
$\Sigma$  & 0 & 0 & 0 & 0 \\
\hline
\end{tabular}
\end{center}

The subscripts $s$, $d$, denote quantities supplied and demanded; The subscript $h$ on the money stock indicates the stock of money held by households, and in this simple model, this is balanced by the amount of money supplied by the government.

Since each row and column sums to zero, there is an \lq adding up constraint' that gives rise to an equation or identity for each one. Column one gives rise to the household budget constraint, wage income less consumption less taxes is equal to the change in money holdings.
$$W.N_s - C_d -T_s = \Delta H_h$$
Column two says that output $Y$ is equal to expenditure (consumption plus government expenditure) and is also equal to income which is  total wages in a labour only economy. Wages are the product of the wage rate $W$ and the number employed $N$.
$$Y = C_s + G_s = W.N_d$$
Column three states that the government's deficit (or surplus) is balanced by the change in the money supply.
$$ G_d -T_d = \Delta H_s $$
The horizontal rows also generate adding up constraints which express the requirement of equality of supply and demand in the various transactions:
\begin{align*}
C_s & = C_d\\
G_s & = G_d\\
N_s & = N_d\\
T_s & = T_d\\
\Delta H_h & = \Delta H_s
\end{align*}
While these constraints must hold for the model to be consistent, they don't contain any information about how these flows are to be brought into equality. To complete the model requires additional equations capturing the relationships between the flow variables. The following two behavioural equations, the tax schedule and the consumption function, will close the model:
\begin{align*}
T_s & = \theta.W.N_S\\
\intertext{and,}
C_d & = \alpha_1.YD + \alpha_2.H_{h-1}\\
\intertext{where $\theta$ is the tax rate, $\alpha_1$ is the propensity to consume out of income, $\alpha_2$ is the propensity to consume out of (last period's) wealth and $YD$ is disposable income,}
YD & = W.N_s -T_s
\end{align*}

The end result is a system of equations with a set of twelve endogenous variables
$\{C_s, C_d, G_s, G_d, N_s, N_d, T_s, T_d, H_h, H_s, Y, YD\}$ (with two of the endogenous variables having lags, $H_h, H_s$); a set of three exogenous variables $\{W, G_s, G_d\}$, 
and three parameters, $\{\theta, \alpha_1, \alpha_2 \}$.

\section{Solutions to the Model}
Despite the fact that the model is so simple that a solution can be found analytically by algebraic simplification, for the purposes of understanding existence and stability conditions, it is useful to manipulate the equations into the form of a linear difference equation:
\newcommand{\vect}[1]{\mathbf{#1}}
$$\vect{x}_t = \vect{A}.\vect{x}_{t-1} + \vect{b}_t$$
where $\vect{x}_t$ is the vector of endogenous variables at time $t$, $\vect{A}$ is a coefficient matrix,
$\vect{x}_{t-1}$ is the vector of lagged variables and $\vect{b}_t$ is a vector of terms that may or may not depend on $t$. In the Linear Difference Equation literature, it is usually called a \emph{forcing term} \cite[pg.~2]{tirelli:2014}. He provides the following cases:
\begin{itemize}
\item When $\vect{b}_t = 0$, the difference equation is said to be homogeneous, and otherwise non-homogeneous
\item When the forcing term is a constant ($b_t = b$ for all $t$), the difference equation is non-homogeneous and \emph{autonomous} (or time-invariant)
\item When $b_t$ is time-dependent the equation is said to be \emph{non-autonomous}; this is the more general situation, allowing the system to capture seasonality, deterministic shocks and other perturbations.
\item The most general form of linear difference equation is one where the coefficient  $\vect{A}$ is also time-varying.
\end{itemize}
To manipulate the equations into linear difference equation form, it is useful to first simplify the model by applying the supply - demand equalities to reduce the number of variables. This results in the following:
\begin{align}
H_s - H_{s-1} + \theta.Y & = G\\
H_h - (1-\alpha_2).H_{h-1} - (1-\alpha_1).(1-\theta).Y & = 0\\
1-\alpha_1.(1-\theta).Y - \alpha_2.H_{h-1} & = G
\end{align}
In the original set of equations the adding up constraint arising from row 6 of the transactions matrix,
$\Delta H_h = \Delta H_s$ has not been used to derive these three equations. 
In \cite[pg.~68]{godleylavoie:2007}  this is described as the redundant equation, if it is included in the equation set, the system would be over-determined. However, it must hold in the final solution.

Equations (1) -- (3) can be written in matrix form as follows:
\begin{equation*}
\begin{pmatrix}
1 & 0 & \theta\\
0 & 1 &  - (1-\alpha_1).(1-\theta)\\
0 & 0 & 1-\alpha_1.(1-\theta)
\end{pmatrix}
\begin{pmatrix}
H_{st}\\
H_{ht}\\
Y_t
\end{pmatrix}
+
\begin{pmatrix}
-1 & 0 & 0\\
0 & - (1-\alpha_2) &  0\\
0 & -\alpha_2 & 0
\end{pmatrix}
\begin{pmatrix}
H_{st-1}\\
H_{ht-1}\\
Y_{t-1}
\end{pmatrix}
=
\begin{pmatrix}
G\\
0\\
G
\end{pmatrix}
\end{equation*}
This has the form
$$\vect{A}.\vect{x}_t + \vect{B}.\vect{x}_{t-1} = \vect{b}$$
If $\vect{A}$ has an inverse, this can be manipulated into the required form as follows:
$$\vect{x}_t = \vect{A}^{-1}.(-\vect{B}).\vect{x}_{t-1} + \vect{A}^{-1}.\vect{b}$$
\subsection{The Stationary State}
If the system reaches a \emph{stationary} state, all variables will be constant from one period to the next, so $\vect{x}_t = \vect{x}_{t-1} = \vect{x}^*$ where $\vect{x}^*$ represents the stationary state solution. In this state it is also the case that 
$H_{h} = H_{h-1}$, $H_{s} = H_{s-1}$ and $Y = Y_{-1}$. So the above equation becomes
$$(\vect{A} + \vect{B}).\vect{x}_{t} = \vect{b}$$
and the stationary state will be given by
$$\vect{x}_{t} = (\vect{A} + \vect{B})^{-1}.\vect{b}$$
provided $\vect{A} + \vect{B}$ has an inverse.
The matrix $\vect{A} + \vect{B}$ is
$$\begin{pmatrix}
0 & 0 & \theta\\
0 & \alpha_2 &   - (1-\alpha_1).(1-\theta)\\
0 & -\alpha_2 & 1-\alpha_1.(1-\theta)
\end{pmatrix}$$
and $|A+B| = 0$. The reason for this is that when the two matrices are added together, the term in $H_s$ cancels out. Equation (1), $H_s - H_{s-1} + \theta.Y  = G$, which says that the government deficit equals the change in the money supply, becomes when  $H_s = H_{s-1}$ a statement that government expenditure equals total taxation, $\theta.Y  = G$. This is expected to be true in the stationary state but leaves the system without a way of evaluating $H_s$. As a work around, the original equation (1) can be swapped with the adding up constraint from the balance sheet $H_s = H_h$. Then the matrix $\vect{A} + \vect{B}$ becomes
$$\begin{pmatrix}
1 & -1 & 0\\
0 & \alpha_2 &   - (1-\alpha_1).(1-\theta)\\
0 & -\alpha_2 & 1-\alpha_1.(1-\theta)
\end{pmatrix}$$
which does have an inverse, and the stationary state solution is
$$\vect{x}^*
=
\begin{pmatrix}
H_s^*\\[5pt]
H_h^*\\[5pt]
Y^*
\end{pmatrix}
=
\begin{pmatrix}
\frac{G(1-\alpha_1).(1-\theta)}{\alpha_2.\theta}\\[5pt]
\frac{G(1-\alpha_1).(1-\theta)}{\alpha_2.\theta}\\[5pt]
\frac{G}{\theta}
\end{pmatrix}$$
For a meaningful stationary state, $H_s^*$ and $H_s^*$ must be positive and non-zero. This will be the case  if $G \geq 0$ and $\alpha_1, \theta < 1$, i.e. if government expenditure is greater than zero and the propensity to consume out of income and the tax rate are less than one. This will be the case in any realistic scenario. $H_s^*$ and $H_s^*$ will have defined values if 
$ \alpha_2 > 0$, i.e. the propensity to consume out of wealth is greater than zero and $\theta>0$, i.e. the tax rate is greater than zero. If the tax rate is zero, from column three of the transactions matrix, $\Delta H_s = G_d$ so the change in the money supply equals government expenditure and the money stock will grow indefinitely. If the propensity to consume out of wealth is zero, then the consumption function becomes
$C_d  = \alpha_1.YD$ and since $\alpha_1<1$ as required above, then $C_d  < YD$ and since
$\Delta H_h = YD - C_d$ it follows that $\Delta H_h$ would always be positive and household cash holdings would grow indefinitely.

In summary, the conditions for a non-zero stationary state are
$$ G \geq 0, \text{\hspace{10pt}} \alpha_1 < 1, \text{\hspace{10pt}} \theta < 0$$
and the conditions for a non-divergent stationary state are
$$ \alpha_2 > 0, \text{\hspace{10pt}} \theta > 0$$.
\subsection{The General Solution}
A general method for the solution of linear difference equations, analogous to the one used for differential equations, is based on the Superposition Principle: the solution of a linear difference equation is
the sum of the solution of its homogeneous part, called the \emph{complementary solution}, and the
\emph{particular integral}. A particular integral is any solution to the non-homogeneous difference equation
$$ \vect{x}_{t} = \vect{x}_{t}^{co} + \vect{x}_{t}^{pi}$$.
For the complementary solution, the homogeneous part is
$$\vect{x}_t = \vect{A}.\vect{x}_{t-1}$$
\cite[]{gandolfo1997} provides the following method...
\begin{center}
...to be continued...
\end{center}

\section{Stability of Solutions}
\begin{center}
...to be continued...
\end{center}

\bibliographystyle{alpha}
\bibliography{biblio}
\end{document}
